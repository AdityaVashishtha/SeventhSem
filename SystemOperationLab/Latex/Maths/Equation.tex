\documentclass{report}
\usepackage{amsmath}
\usepackage{amssymb}
\usepackage{ragged2e}
\usepackage{mathpazo}
\usepackage{mathtools}
\usepackage{algorithmic}
\usepackage{minted}
\usepackage{hyperref}
\hypersetup{
    colorlinks,
    linkcolor=black
}
% custom commands --------------------
\newcommand{\eqname}[1]{\tag*{#1}}
\newcommand{\spc}{\hspace{1cm}}

\newcommand{\boxim}{\rule{2em}{2em}}


\begin{titlepage}
    \title{An Assignment on \LaTeX}    
    \author{Aditya Vashishtha}
    \date{\today}
\end{titlepage}


\begin{document}
    \maketitle
    \tableofcontents
    \chapter{Tables}
        \section{Dummy section} \label{sec:1}
        \subsection{Dummy subsection}
        Lorem Ipsum is simply {\footnote{This is first footnote.}} dummy text of the {printing and typesetting industry.}  \rm Lorem Ipsum has been the industry's standard dummy text ever since the 1500s, when an unknown printer took a galley of type and scrambled it to make a type specimen book. It has survived not only five centuries, but also the leap into electronic typesetting, remaining essentially unchanged. It was popularised in the 1960s with the release of Letraset sheets containing Lorem Ipsum passages, and more recently with desktop publishing software like Aldus PageMaker including versions of Lorem Ipsum.
        \subsection{Second subsection}
        Lorem Ipsum is simply \footnote{This is second footnote.} dummy text of the {printing and typesetting industry.}  \rm Lorem Ipsum has been the industry's standard dummy text ever since the 1500s, when an unknown printer took a galley of type and scrambled it to make a type specimen book. It has survived not only five centuries, but also the leap into electronic typesetting, remaining essentially unchanged. It was popularised in the 1960s with the release of Letraset sheets containing Lorem Ipsum passages, and more recently with desktop publishing software like Aldus PageMaker including versions of Lorem Ipsum.
        \section{Second section}
        \subsection{Dummy subsection}
        Lorem Ipsum is simply dummy text of the {printing and typesetting industry.}  \rm Lorem Ipsum has been the industry's standard dummy text ever since the 1500s, when an unknown printer took a galley of type and scrambled it to make a type specimen book. It has survived not only five centuries, but also the leap into electronic typesetting, remaining essentially unchanged. It was popularised in the 1960s with the release of Letraset sheets containing Lorem Ipsum passages, and more recently with desktop publishing software like Aldus PageMaker including versions of Lorem Ipsum.
        \subsection{Second subsection}
        Lorem Ipsum is simply dummy text of the {printing and typesetting industry.}  \rm Lorem Ipsum has been the industry's standard dummy text ever since the 1500s, when an unknown printer took a galley of type and scrambled it to make a type specimen book. It has survived not only five centuries, but also the leap into electronic typesetting, remaining essentially unchanged. It was popularised in the 1960s with the release of Letraset sheets containing Lorem Ipsum passages, and more recently with desktop publishing software like Aldus PageMaker including versions of Lorem Ipsum.
    
    \chapter{Math}
    \section{Quadratic}
    \subsection{Que.3 Solve the equation $5x^2 - 2x + 1 = 0  $. }
        \normalsize
        \textbf{solution:} \\
        \begin{align*}
            5x^2-2x+1 = 0 {\eqname{...given}}
        \end{align*}
        \flushleft
        \textbf{Using formula :} \\
        $x_1,x_2$ root of equation \\
        $ax^2+bx+c=0$ are, \\
        $$ x_1,x_2 = \frac{-b \pm \sqrt{b^2 - 4ac}}{2a} $$
        { here for given equation,
        $  a = 5 , b = -2 , c = 1 $ \\ we have, } \\
        \begin{align*}
            x_1,x_2 &= \frac{-(-2) \pm \sqrt{ (-2)^2 - 4 * (5) * (1)  }}{2*(5)} \\ \\
            x_1,x_2 &=\frac{ 2 \pm \sqrt{ 4 - 20  }}{10} \\ \\
            x_1,x_2 &=\frac{ 2 \pm \sqrt{ -16  }}{10} \\ \\
            x_1,x_2 &=\frac{ 2 \pm 4*\sqrt{ -1  }}{10} \\ \\
            \Rightarrow x_1, x_2 &=\frac{ 2 \pm 4i}{10} \eqname{\bf Ans} \\ \\
        \end{align*}
    
    \section{Progression}
    \subsection{Summation of $n^2$ }
        \large
        \begin{align*}
        \sum_{x=1}^{x=n} x^2 &= 1^2 + 2^2 + 3^3 + 4^2 ... + (n-1)^2 + n^2 \\
        \sum_{x=1}^{x=n} x^2 &= \frac{(n)(n+1)(2n+1)}{6}
        \end{align*}
    
    \section{Matrix}
    \subsection{Multiplication of Matrix}
        \Large
        Find X*Y if,
        \large
        $$
        X = 
        \begin{bmatrix}
            1 & 2 \\
            3 & 4 \\
            5 & 6 
        \end{bmatrix}
        , Y =
        \begin{bmatrix}
            7 & 8 & 9 & 0 \\
            1 & 2 & 3 & 4
        \end{bmatrix}
        $$
        $$
        XY = 
        \begin{bmatrix}
            1 & 2 \\
            3 & 4 \\
            5 & 6 
        \end{bmatrix}
        \begin{bmatrix}
            7 & 8 & 9 & 0 \\
            1 & 2 & 3 & 4
        \end{bmatrix}
        $$

        $$
         XY =
        \begin{bmatrix}
            (1*7 + 2*1) & (1*8+2*2) & (1*9+2*3) & (1*0+2*4) \\
            (3*7 + 4*1) & (3*8+4*2) & (3*9+4*3) & (3*0+4*4) \\
            (5*7 + 6*1) & (5*8+6*2) & (5*9+6*3) & (5*0+6*4) 
        \end{bmatrix}
        $$


    \section{Calculas}
    \subsection{Integration Problem}
        \Large
            Solve the Integration $ \int xcos(x^2)dx $. \\
        \large
        \textbf{solution:} \\        
        \begin{align*}
            f(x) &= \int{x*cos(x^2)dx} \tag{given} \\
                 &= \int {x*cos(x^2)dx} \\
            putting, \spc x^2 &= t \tag{1} \\
        \end{align*}
        differenting both the side equation 1,
        \begin{align*}\label{eq:1}
            x^2 dx &= t dx \\
            2x &= t \frac{dt}{dx} \\
            \Rightarrow x*dx &= \frac{t}{2}dt \tag{2}
        \end{align*}
        putting value of equation 2\textsuperscript{nd} in given equation, \\ 
        \begin{align*}
            f(t) &= \int{cos(t)dt} \\
            f(t) &= sin(t) + C \spc \spc \spc \because \int cos(x) dx = sin(x) + c
        \end{align*}
        as\footnote{$\because \int cos(x) = sin(x) + c$} putting, 
        $$ t=\sqrt x $$
         from equation 1 \\
        we have 
        \begin{align*}
            f(x) &= \int xcos(x^2)dx = sin(\sqrt x) + C  \\
            f(x) &= sin(\sqrt x) + C  \tag{Ans} \\
        \end{align*}
    \newline 
    \newline
    \section{Binomial}
        \Large
        Write binomial formula for $(a + b)^5$.\\
        \large
        $$ (a+b)^5 = \binom{5}{0}a^5b^0 + \binom{5}{1}a^4b^1 + \binom{5}{2}a^3b^2 + \binom{5}{3}a^2b^3  \binom{5}{4}a^1b^4 + \binom{5}{5}a^0b^5 $$

    \chapter{Algorithm}
        \section{Sorting}
        \subsection{Bubble sort}
        \textbf{Input:} list of elements.\\
        \textbf{Output:} Sorted list.\\
        \textbf{BubbleSort(List)} \\
        \begin{algorithmic}[1] \label{algo:1}
            \STATE set $n \leftarrow length(list)$
            \FOR{$i=0$ \TO $n-1$}
            \FOR{$j=0$ \TO $(n-i-1)$}
            \STATE set $ swapped \leftarrow false $
            \IF {$ list[j-1] > list[j] $}
            \STATE $swap(list[j-1],list[j])$
            \STATE set $ swapped \leftarrow true $
            \ENDIF
            \ENDFOR
            \IF{ ($swapped == false$ )}
            \STATE $break$
            \ENDIF
            \ENDFOR
        \end{algorithmic}

        \chapter{References}
            \begin{equation} 
                x =y
            \end{equation}
            First equation \ref{eq:1}
            \\
            First Section \ref{sec:1}
            \\
            BubbleSort refered in \ref{algo:1}
\end{document}